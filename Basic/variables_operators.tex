\subsection{Exercise: variables and operators}

\lettrine{C}{reate} variables with names of your choice, holding the values \code{1}, \code{2.55} and \code{"1"}. Print out all variables and their types. 

Create a new variable, which is the sum of the first two variables. Print out this new variable and its type.

Try adding the second and third variable in the same way. Does this work? Why not?\footnote{More information on error messages, or `tracebacks', will be given in the section on Tracebacks.}

Using a variable \code{r} to represent the radius of a circle, have Python print out the circle's circumference and area. For this exercise, you may assume that $\pi = 3$ (or feel free to use a closer approximation). You may choose the value of \code{r} yourself.

\begin{hightask}[Task]
	Try writing a series of expressions in a code which will calculate the solutions to a quadratic equation of the form,
	\begin{equation}
	0 = a\; x^2 + b\; x + c
	\end{equation}
	by evaluating the quadratic formula,
	\begin{equation}
	x = \frac{- b \pm \sqrt{b^2 - 4\; a\; c}}{2 \; a}
	\end{equation}
\end{hightask}

% Ideally, I'd like to add one or two exercises on comparisons and/or integer division and modulus, but I can't come up with any on the spot.

